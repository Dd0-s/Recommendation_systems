\documentclass[a4paper,12pt]{article}
\usepackage[T2A]{fontenc}
\usepackage[utf8]{inputenc}	
\usepackage[english,russian]{babel}	
\usepackage{amsmath,amsfonts,amssymb,amsthm,mathtools} 
\usepackage[left=16mm,  top=15mm,  right=16mm,  bottom=20mm, nohead, nofoot]{geometry}
\usepackage{graphicx}
\graphicspath{{pictures/}}
\DeclareGraphicsExtensions{.pdf,.png,.jpg}
\usepackage{wasysym}
\usepackage{tikz}
\usepackage[unicode, pdftex]{hyperref}

\begin{document} 

\begin{center}
    \textbf{Лабораторная работа 2}\\
    \textbf{Богданов Александр, Б05-003}
\end{center}

Запишем информацию о ручках:

\begin{center}
    \begin{tabular}{|c|c|c|}
        \hline
        \text { Ручка } & \text { Средння награда } & \text { Количество использований } \\
        \hline
        1 & 4.6 & 181 \\
        2 & 4.3 & 21 \\
        3 & 4.7 & 384  \\
        \hline
    \end{tabular}
\end{center}

Общее количество использований: t = 586.

\section{Задача}

\subsection{Задача}
    \textbf{Условие:} Найдите $\varepsilon$-жадная стратегию.
    $\pi_{\varepsilon}$ (положите $\varepsilon$ = 0.01).

    \begin{align*}
        a_t = 
        \begin{cases}
            \arg \max\limits_{a \in A} Q(a) & \text{с вероятностью}\ \varepsilon \\
            \text{random} (A) & \text{с вероятностью}\ 1 - \varepsilon
        \end{cases}
    \end{align*}

    В $\varepsilon$ случаях выбирается случайная ручка, в $1 - \varepsilon$ выбирается оптимальная ручка, тогда $\varepsilon$-жадная политика: $\pi_{\varepsilon} = \left[ \frac{\varepsilon}{3}; \frac{\varepsilon}{3}; \frac{\varepsilon}{3} + 1 - \varepsilon \right] = \left[ \frac{\varepsilon}{3}; \frac{\varepsilon}{3}; 1 - \frac{2 \varepsilon}{3} \right] = [0.0033, 0.0033, 0.9933]$.

\subsection{Задача}
    \textbf{Условие:} Найдите UCB стратегию $\pi_\text{UCB}$ ($\alpha = 0.5$).

    \begin{align*}
        a_t = \arg \max_{a \in A} Q(a) + \alpha \sqrt{\frac{\log t}{N_t (a)}}
    \end{align*}

    \begin{center}
        \begin{tabular}{|c|c|c|c|}
            \hline
            \text { Ручка } & \text { Средняя награда (отнормированная) } & \text { Добавочная константа } & \text { Сумма }\\
            \hline
            1 & 0.920 & 0.094 & 1.014\\
            2 & 0.860 & 0.275 & 1.135\\
            3 & 0.980 & 0.064 & 1.044\\
            \hline
        \end{tabular}
    \end{center}

    То есть UCB политика: $\pi_\text{UCB} = [0, 1, 0]$.

\subsection{Задача}
    \textbf{Условие:} Что нужно чтобы применить здесь томсоновское сэмплирование?

    \begin{enumerate}
        \item Ввести априорное распределение.
        \item Уметь к априорному распределению строить сопряженное. Для бинарного автомата обычно берут $beta$ распределение. Наш автомат не бинарный, но можно провести бинаризацию: 1, 2, 3 -неудача; 4, 5 - успех.
    \end{enumerate}

\section{Задача}

\subsection{Задача}
    \textbf{Условие:} посчитайте logging policy $\pi_0$.

    Оценим частотной оценкой: $\pi_0 = [0.309, 0.036, 0.655]$.

\subsection{Задача}
    \textbf{Условие:} Oцените стратегию $\pi_1$ = [0.3, 0.04, 0.66].

    \begin{align*}
        \hat{V} (\pi_1, D) = \mathbb{E}_{p(x) \pi_1 (a|x) p(r|x, a)}[r] = \mathbb{E}_{\pi_1 (a) p(r|a)}[r] = \sum_{a, r} r p(r|a) \pi_1 (a) = \sum_a \mathbb{E} [r|a] \pi_1 (a)
    \end{align*}

    \begin{align*}
        \hat{V} (\pi_1, D) = 0.3 * 4.6 + 0.04 * 4.3 + 0.66 * 4.7 = 4.654 
    \end{align*}

\subsection{Задача}
    \textbf{Условие:} Oцените стратегию $\pi_2$ = [0.3, 0.66, 0.04].

    \begin{align*}
        \hat{V} (\pi_2, D) = \mathbb{E}_{p(x) \pi_2 (a|x) p(r|x, a)}[r] = \mathbb{E}_{\pi_2 (a) p(r|a)}[r] = \sum_{a, r} r p(r|a) \pi_2 (a) = \sum_a \mathbb{E} [r|a] \pi_2 (a)
    \end{align*}

    \begin{align*}
        \hat{V} (\pi_2, D) = 0.3 * 4.6 + 0.66 * 4.3 + 0.04 * 4.7 = 4.406 
    \end{align*}

\subsection{Задача}
    \textbf{Условие:} Oцените стратегию $\pi_{\varepsilon}$ и $\pi_\text{UCB}$. 

    \begin{align*}
        \hat{V} (\pi, D) = \mathbb{E}_{p(x) \pi (a|x) p(r|x, a)}[r] = \mathbb{E}_{\pi (a) p(r|a)}[r] = \sum_{a, r} r p(r|a) \pi (a) = \sum_a \mathbb{E} [r|a] \pi (a)
    \end{align*}
        
    \begin{enumerate}
        \item Так как политика $\pi_{\varepsilon}$ не меняется, то можно оценить как в предыдущих задачах:

        \begin{align*}
            \hat{V} (\pi, D) = 0.0033 * 4.6 + 0.0033 * 4.3 + 0.9933 * 4.7 = 4.698 
        \end{align*}

        \item Политика $\pi_\text{UCB}$ = [0, 1, 0] после некоторого количества использований поменяется на $\pi_\text{UCB}$ = [0, 0, 1], поэтому ее можно оценить так:

        \begin{align*}
            4.3 \leq \hat{V} (\pi, D) \leq 4.7
        \end{align*}
        
    \end{enumerate}
    
\subsection{Задача}
    \textbf{Условие:} Проанализируйте результаты. Возможно ли оценить стратегии из 3 предыдущих пунктов с адекватной точностью?

    $\pi_1$, $\pi_2$, $\pi_{\varepsilon}$, можно оценить явно посчитав. $\pi_\text{UCB}$ можно оценить интервалом -- интервал не очень большой, поэтому оценка достаточно точная.

\section{Задача}

\subsection{Задача}
    \textbf{Условие:} Докажите что оценивание стратегий через IPS несмещенное.

    \begin{align*}
        \hat{V}_{\text{IPS}} (\pi_{test}, D) = \frac{1}{n} \sum_{i = 1}^n \frac{\pi_{test} (a_i, x_i)}{\pi_0 (a_i, x_i)} \cdot r_i, 
    \end{align*}

    где

    \begin{align*}
        D = \{ (x_i, a_i, r_i)\}_{i = 1}^n \sim \prod_{i = 1}^n p (x_i) \pi_0 (a_i|x_i) p(r_i|x_i, a_i)
    \end{align*}

    \begin{align*}
        \mathbb{E}_D \left[ \hat{V}_{\text{IPS}} (\pi_{test}, D) \right]
        &=
        \frac{1}{n} \sum_{i = 1}^n \int p (x_i) \pi_0 (a_i|x_i) p(r_i|x_i, a_i) \frac{\pi_{test} (a_i, x_i)}{\pi_0 (a_i, x_i)} r_i \\
        &=
        \int p (x_i) \pi_{test} (a_i, x_i) p(r_i|x_i, a_i) r_i\\
        &=
        \mathbb{E}_{p(x) \pi_{test} (a|x) p(r|x, a)}[r] = V(\pi_{test})
    \end{align*}


\subsection{Задача}
    \textbf{Условие:} При каких необходимых условиях выполняется несмещенность?

    Если $\pi_0$ может генерировать все действия, которые может сгенерировать $\pi_{test}$, иначе будет интеграл не по всему множеству.

\end{document}
